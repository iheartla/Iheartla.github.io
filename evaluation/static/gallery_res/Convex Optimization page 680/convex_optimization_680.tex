\documentclass[12pt]{article}
\usepackage{mathdots}
\usepackage[bb=boondox]{mathalfa}
\usepackage{mathtools}
\usepackage{amssymb}
\usepackage{libertine}
\DeclareMathOperator*{\argmax}{arg\,max}
\DeclareMathOperator*{\argmin}{arg\,min}
\usepackage[paperheight=8in,paperwidth=4in,margin=.3in,heightrounded]{geometry}
\let\originalleft\left
\let\originalright\right
\renewcommand{\left}{\mathopen{}\mathclose\bgroup\originalleft}
\renewcommand{\right}{\aftergroup\egroup\originalright}
\begin{document}

\begin{center}
\resizebox{\textwidth}{!} 
{
\begin{minipage}[c]{\textwidth}
\begin{align*}
 \omit \span \begin{bmatrix}
\textit{P₁} & 0\\
0 & \textit{P₃}\\
\end{bmatrix}\begin{bmatrix}
\mathit{L} & 0\\
{\textit{P₃}}^T\mathit{C}{\textit{P₂}}^T\mathit{U}^{-1} & -\textit{L̃}\\
\end{bmatrix}\begin{bmatrix}
\mathit{U} & \mathit{L}^{-1}{\textit{P₁}}^T\mathit{B}\\
0 & \textit{Ũ}\\
\end{bmatrix}\begin{bmatrix}
\textit{P₂} & 0\\
0 & I_{ \mathit{n} }\\
\end{bmatrix} \\
\intertext{where} 
\textit{P₁} & \in \mathbb{R}^{ \mathit{m} \times \mathit{m} } \\
\textit{P₂} & \in \mathbb{R}^{ \mathit{m} \times \mathit{m} } \\
\textit{P₃} & \in \mathbb{R}^{ \mathit{n} \times \mathit{n} } \\
\mathit{B} & \in \mathbb{R}^{ \mathit{m} \times \mathit{n} } \\
\mathit{C} & \in \mathbb{R}^{ \mathit{n} \times \mathit{m} } \\
\mathit{L} & \in \mathbb{R}^{ \mathit{m} \times \mathit{m} } \\
\textit{L̃} & \in \mathbb{R}^{ \mathit{n} \times \mathit{n} } \\
\mathit{U} & \in \mathbb{R}^{ \mathit{m} \times \mathit{m} } \\
\textit{Ũ} & \in \mathbb{R}^{ \mathit{n} \times \mathit{n} } \\
\\
\end{align*}
\end{minipage}
}
\end{center}

\end{document}
