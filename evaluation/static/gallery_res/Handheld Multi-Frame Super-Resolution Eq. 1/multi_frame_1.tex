\documentclass[12pt]{article}
\usepackage{mathdots}
\usepackage[bb=boondox]{mathalfa}
\usepackage{mathtools}
\usepackage{amssymb}
\usepackage{libertine}
\DeclareMathOperator*{\argmax}{arg\,max}
\DeclareMathOperator*{\argmin}{arg\,min}
\usepackage[paperheight=8in,paperwidth=4in,margin=.3in,heightrounded]{geometry}
\let\originalleft\left
\let\originalright\right
\renewcommand{\left}{\mathopen{}\mathclose\bgroup\originalleft}
\renewcommand{\right}{\aftergroup\egroup\originalright}
\begin{document}

\begin{center}
\resizebox{\textwidth}{!} 
{
\begin{minipage}[c]{\textwidth}
\begin{align*}
\textit{C(x,y)} & = \frac{\sum_\mathit{n} \sum_\mathit{i} \mathit{c}_{\mathit{n}, \mathit{i}} \cdot \mathit{w}_{\mathit{n}, \mathit{i}} \cdot \textit{R̂}_{ \mathit{n} }}{\sum_\mathit{n} \sum_\mathit{i} \mathit{w}_{\mathit{n}, \mathit{i}} \cdot \textit{R̂}_{ \mathit{n} }} \\
\intertext{where} 
\mathit{c} & \in \mathbb{R}^{ \mathit{f} \times \mathit{s} } \text{ the value of the Bayer pixel} \\
\mathit{w} & \in \mathbb{R}^{ \mathit{f} \times \mathit{s} } \text{ the local sample weight} \\
\textit{R̂} & \in \mathbb{R}^{ \mathit{f}} \text{ the local robustness} \\
\\
\end{align*}
\end{minipage}
}
\end{center}

\end{document}
